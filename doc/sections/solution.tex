%!TEX root=../document.tex

\section{Kompetenzen}
\label{sec:Kompetenzen}
\subsection{Dezentrale Systeme}
\label{subsec:Dezentrale Systeme}
\begin{enumerate}
\item Können Lastverteilung auf
Applikationsebene realisieren
\item Können Sicherheitskonzepte für
verteilte, dezentrale Systeme
entwickeln
\item Können ausfallsichere replizierte Datenbanksysteme und dezentrale Systeme installieren, warten und entwerfen
\item Können verteilte und redundante
Dateisysteme einsetzen
\end{enumerate}

\subsection{Systemintegration und Infrastruktur}
\label{subsec:Systemintegration und Infrastruktur}
\begin{enumerate}
\item Können
Virtualisierungstechniken
beschreiben und diese im
Unternehmen geeignet einsetzen
\item Können Mechanismen zur
Softwareverteilung
implementieren
\item Können in Unternehmensnetzwerken ausfallsichere und redundante informationstechnische Systemarchitekturen mit unterschiedlichen Betriebssystemen realisieren
\item Können Fernwartungstechniken
beschreiben und diese im
Unternehmen geeignet umsetzen
\end{enumerate}

\section{Cloud Computing und Internet of Things}
\label{sec:Cloud Computing und Internet of Things}
\subsection{Systemarchitektur und Cloud Anbieter}
\label{subsec:Systemarchitektur und Cloud Anbieter}
Was ist eine Cloud?

Welche Anbieter gibt es und welche wäre hier am geeignetsten? (AWS, Azure, Google, ...?)


Als Alternative selber Hosten?

Wenn ja, welche leistungstechnischen Dimensionen müsste die annehmen?

Was ist auf lange Sicht besser?
(in Bezug auf Kosten, Wartung, Verfügbarkeit, Skalierbarkeit, ...)

Welcher Application Stack, der Load Balancing, Hochverfügbarkeit und eine automatische Skalierung ermöglichen soll, soll verwendet werden?

\subsection{Kommunikation}
\label{subsec:Kommunikation}
Was ist P2P und wie funktioniert es?

Wie funktioniert die Client-Server Architektur?

Vergleich beider Architekturen

Wieso P2P?

Idee(n) zur konkreten Umsetzung

\subsection{Fernwartung der Bojen}
\label{subsec:Fernwartung der Bojen}
Fernwartung via SSH

Einsatz einer Administrationssoftware zum Einspielen von Updates, Erstellen und Einspielen von Backups, Überprüfen des Systemstatus ("Health Check"), ...
(-> Performanceprobleme?)

Namensgebung, Einsatz von DNS

\subsection{Verteilte und redundante Dateisysteme}
\label{subsec:Verteilte und redundante Dateisysteme}
Was ist ein verteiltes Dateisystem?

Welche Vorteile bietet ein verteiltes Dateisystem?

Welche Implementierungen gibt es und welche würde sich besonders anbieten (Vergleich von HadoopFS, GlusterFS und OCFS2?)


\section{Automatisierung, Regelung und Steuerung}
\label{sec:Automatisierung, Regelung und Steuerung}
\subsection{Scripting}
\label{subsec:Scripting}
Automatisieren von bestimmten Vorgängen mittels Shell-Scripts und Cronjobs? (z.B. Sammeln und Auswerten der Sensordaten)

???

Verwenden eines Caches für Softwarepakete und Konfigurationsdateien (siehe Anmerkungen auf der letzten Seite)


\section{Security, Safety, Availability}
\label{sec:Security, Safety, Availability}
Einsatz einer Firewall?

Vergleich von Hardware und Software Firewalls

Firewall load balancen?

SSH-Zugriff absichern (Key-based authentification, Verschlüsselter Traffic, Port ändern, Anzahl der fehlgeschlagenen Loginversuche pro IP limitieren, ...)

Schutz vor DDoS-Attacken

Was ist Port Trunking?

Welche Vorteile ergeben sich daraus?

Wie kann es hier praktisch angewandt werden?

(-> Verfügbarkeit?)

Verwendung von Containern auf Client- und Serverseite

Verwendung Kubernetes auf Serverseite? (doch eher zu Internet of Things?) -> Hochverfügbarkeit mittels Kubernetes Clustern?

\section{Authentication, Authorization, Accounting}
\label{sec:Authentication, Authorization, Accounting}
Benutzerverwaltung mittels eines Verzeichnisdienstes (LDAP)

Was ist LDAP?

Aufbau von LDAP

Verwalten von Benutzern?

LDAP load balancen und replizieren (???)

Verwendung eines Key Distribution Centers und Single Sign On zwecks sicherer und einfacherer Benutzeranmeldung und Benutzerverwaltung

\section{Desaster Recovery}
\label{sec:Desaster Recovery}
Verwendung von zentralen Client Images, im Falle eines Rollbacks

Plan zur Verteilung dieser Images

Anlegen und Verwahren von Backups (wann, wo, wie und wie oft?)

Bildung eines Clusters, um die Möglichkeit zum Bilden eines Desasters zu minimieren

Planen und Umsetzen eines Recovery Plans


\section{Algorithmen und Protokolle}
\label{sec:Algorithmen und Protokolle}
Heartbeat (Checken, ob Boje noch lebt) +  konkreter Einsatz

???



\section{Konsistenz und Datenhaltung}
\label{sec:Konsistenz und Datenhaltung}
Verwendung von Samba und NFS

Was ist es?

Was kann es?

Setup und konkrete Anwendung

Verwaltung von NFS, Userrechte -> LDAP

Speichern von großen Datenmengen in Rechenzentren in Clustern

Evaluieren von geeigneten Dateisystemen für solche Cluster


\section{Anmerkungen}
\label{sec:Anmerkungen}
Die Bojen sollten nach Möglichkeit nicht direkt von Außerhalb erreichbar sein.


%\begin{figure}[!h]
%	\begin{center}
%		\includegraphics[width=1.0\linewidth]{images/grafik.png}
%		\caption{Catopn \cite{Quelle}}
%		\label{Label}
%	\end{center}
%\end{figure}

%\begin{lstlisting}[frame=single, language=bash, caption=Caption]
%
%\end{lstlisting}
